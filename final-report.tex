%%%%%%%%%%%%%%%%%%%%%%%%%%%%%%%%%%%%%%%%%
% University Assignment Title Page LaTeX Template Version 1.0 (27/12/12)
%
% This template has been downloaded from: http://www.LaTeXTemplates.com
%
% Original author: WikiBooks (http://en.wikibooks.org/wiki/LaTeX/Title_Creation)
%
% License: CC BY-NC-SA 3.0 (http://creativecommons.org/licenses/by-nc-sa/3.0/)
% 
% Instructions for using this template: This title page is capable of being compiled as is. This is not useful for
% including it in another document. To do this, you have two options: 
%
% 1) Copy/paste everything between \begin{document} and \end{document} starting at \begin{titlepage} and paste this into
% another LaTeX file where you want your title page.  OR 2) Remove everything outside the \begin{titlepage} and
% \end{titlepage} and move this file to the same directory as the LaTeX file you wish to add it to.  Then add
% \input{./title_page_1.tex} to your LaTeX file where you want your title page.
%
%%%%%%%%%%%%%%%%%%%%%%%%%%%%%%%%%%%%%%%%%
%\title{Title page with logo} ----------------------------------------------------------------------------------------
%PACKAGES AND OTHER DOCUMENT CONFIGURATIONS
%----------------------------------------------------------------------------------------

\documentclass[12pt]{article}
\usepackage[english]{babel}
\usepackage[utf8x]{inputenc}
\usepackage{amsmath}
\usepackage{graphicx}
\usepackage{svg}
\usepackage[colorinlistoftodos]{todonotes}
\usepackage{listings}
\usepackage{tabularx}

\begin{document}

\begin{titlepage}

\newcommand{\HRule}{\rule{\linewidth}{0.5mm}} % Defines a new command for the horizontal lines, change thickness here

\center % Center everything on the page
 
%----------------------------------------------------------------------------------------
% HEADING SECTIONS
%----------------------------------------------------------------------------------------

\textsc{\LARGE Nanyang Technological University}\\[1.5cm] % Name of your university/college
\textsc{\Large School of Computer Engineering}\\[0.5cm] % Major heading such as course name
\textsc{\large Final Year Project}\\[0.5cm] % Minor heading such as course title

%----------------------------------------------------------------------------------------
%	TITLE SECTION
%----------------------------------------------------------------------------------------

\HRule \\[0.4cm] { \huge \bfseries Final Report}\\[0.4cm] % Title of your document \HRule \\[1.5cm]
{ \huge \bfseries Android application for cloud-based healthcare monitor}\\[0.4cm] % Title of your document \HRule \\[1.5cm]
 
%---------------------------------------------------------------------------------------- AUTHOR SECTION
%----------------------------------------------------------------------------------------

\begin{minipage}{0.4\textwidth}
    \begin{flushleft}
        \large \emph{Author:}\\ Tuan Phong
        \textsc{Nguyen} % Your name
    \end{flushleft}
\end{minipage} ~
\begin{minipage}{0.4\textwidth}
    \begin{flushright}
        \large \emph{Supervisor:} \\ Assoc Prof Chiew Tong
        \textsc{Lau} % Supervisor's Name \end{flushright} \end{minipage}\\[2cm]
    \end{flushright}
\end{minipage}

% If you don't want a supervisor, uncomment the two lines below and remove the section above
%\Large \emph{Author:}\\ John \textsc{Smith}\\[3cm] % Your name

%---------------------------------------------------------------------------------------- DATE SECTION
    %----------------------------------------------------------------------------------------

{\large \today}\\[2cm] % Date, change the \today to a set date if you want to be precise

%---------------------------------------------------------------------------------------- LOGO SECTION
%----------------------------------------------------------------------------------------

%----------------------------------------------------------------------------------------

\vfill % Fill the rest of the page with whitespace
\end{titlepage}

\begin{abstract}
There is an emerging need for proactive healthcare monitor that is enabled by the development of portable devices and
Internet of Things. Based on the related work on the network of sensor devices to collect medical data, this project
aims to complete the solution by implementing a layer that allows users to store, view and share their data in a reliable
and distributed cloud-based system. In this project, a backend system hosted on cloud computing service and an android
mobile client was developed. Google App Engine was selected as the platform for server due to its low cost and reliable
media to connect to large number of users. On the other hand, the android application was completed as a proof of concept
of how the data can be presented to the users in a friendly and convenient way while following the latest guidelines
in android development by Google.
\end{abstract}

\newpage
\tableofcontents

\linespread{1.25}

\section{Introduction}
Along with the increasing popularity of compact and portable platform for measuring health related data, there are many
products on the market that are capable of reading and sending data to a computer system. These devices are often built
ready for communicating with mobile devices via Bluetooth to exchange data so that those data can be displayed visually.
However, a universal platform to collect and display the data to users has not been implemented yet. On the other hand,
the measured data is not frequently recorded so that it can be used as reference in diagnosing or study. In addition,
many hospitals and clinics have to face the challenge of storing and managing the huge amount of medical data of patients
in different forms that are not efficient and reliable. There is such a need to create a platform where the medical data
of patients can be stored and viewed easily by patients and doctors. This project aims to implement such a platform with
consideration on capability of adapting multiple types of devices, data sharing as well as the scalability.  The scope
of this project is to implement a cloud-based working application that consists of a backend server for storing data
and an Android application that allows users to query their own and other users' data. Overall, the scope of the
application can be described with the following features:
\begin{enumerate}
    \item Users are able to post or submit the medical record (data) to a persistent database (database)
    \item Users are able to view their own data as numerical listing and graphical representation
    \item Users are able to request to view other users' data and view all authorized data sets simultaneously
    \item Users are able to accept a request to view their own data or cancel an authorized request
    \item Users are able to receive notifications on their phone about certain events such as emergency or critical
        data timely.
\end{enumerate}

\section{Related work}
In the last decade, it is observed that healthcare is shifting from traditional reactive approach to proactive approach so
that the issues can be discovered and treated at an early stage. This approach requires a solution to continuously
collect sensor data and monitor through a reliable service.  With the development of technology and medical system,
there are several studies to create a system to monitor patients? condition in real-time effectively and correctly to
enable proactive monitoring \cite{5076774} with the help of embedded computer system \cite{4062461}. Furthermore,
devices with communication capabilities have been developed to measure human body's reading for long-range medical
healthcare system and avoid the cost of nursing expense in families \cite{5918041}. On the other hand, the system which
provide a common interface for large number of users must be able to serve great amount of traffics and interactions
with the clients.  In order to handle large volume of simultaneous interaction to the database, an efficient and
scalable approach must be used for database handling instead of traditional server. A highly available and reliable
distributed platform is provided by Google Datastore, which offers many possibilities in developing such a system to
store the data on the cloud \cite{7059154}.  Superior communication between healthcare professionals is now possible
thanks to the advancement of wireless sensors and Internet of Things. Such a development helps the healthcare system to
provide a better service by allowing the scientists to access the bigger sample size and the users to utilize the
collected sensor data without expert knowledge \cite{7027488}.  To extend the advancement to ubiquitous home
healthcare, array of sensor devices to detect Electrocardiogram (ECG) and Photoplethysmography (PPG), blood glucose
and ear temperature was developed for Home Integrated Health Monitoring in \cite{4352299} and in \cite{5918041}. Data
are collected from the devices around the house via ZigBee communication technology, and from those devices, transmitted
to home server and then medical institutions. However, viewing and interpretation of the data by the users is still not
properly implemented.  On the other hand, cloud computing is an emerging trend that offers several advantages to the
traditional physical servers and hosting. Google Cloud Computing platform can reduce the cost of large-scale clusters
and fault tolerance server to extremely low \cite{5718338}. The solution for healthcare monitor can be implemented
based on Google Cloud Computing stack and use its messaging service, Google Cloud Messaging to send the notifications to
the large number of client devices. Although in real world application, the Google Cloud Messaging are observed to have
unpredictable time delay of message arrival, it is capable of delivering messaging to significantly large number of
subscribers in a reasonable timeframe. Hence, Google Cloud computing and google Cloud Messaging is a suitable
infrastructure for an application that have random accesses by large user base \cite{7037233}.

\section{Architecture design}
\label{sec:Architecture design}
The application contains 3 main components namely the client, the backend server and the database.

\subsection{Client side}
The client is responsible for displaying the data acquired from the physical measuring devices visually as graph or
reading listing. The client application needs to acquire the data records from the backend server through network
communication and cache it in local database for presenting to the app user. In order to server as a universal interface to
various types of medical record, the interface is capable of displaying different types of graph for different record
type and designed to be extended when there is need for a new representation of data. The client connects 
to the back-end to interact with other users of the application to share and access their data in the database. The
client also provides the user with an interface to search for another users of the application and interact with them.
Users can subscribe to another user to view their data and receive notifications upon certain events are triggered such
as when new data is uploaded or when the reading is beyond safe range. As the database is accessed through a RESTful API
via HTTP request and response, the client application can be implemented in any platform e.g. Android, iOS, native
desktop program or web application. Previously, there were many medical monitor devices that are capable of connecting
to smart phones via Bluetooth and hence required such a platform to posting the data upstream. But with the introduction
of various small-size computing module that is network capable (Bluetooth, Ethernet) such as Raspberry Pi 3 or Intel
Edison, the link between measuring devices and the server can be completed easily with an affordable cost. Due to time
constraint and man power of this project as well as the extensibility to other projects, the client is only implemented
for Android phones.

\subsection{Server side}
The server is a Web server that exposes a RESTful API to the clients and serves the incoming requests. Such a RESTful
API that communicates via HTTP request and response is a standard way for network application and provides an
abstraction layer between front-end and back-end, allowing a wide array of client devices to communicate with the
server using standard message. To perform the business logic, the server interacts with a persistent database to provide
the service as application backend.  The server side is chosen to be hosted on Google App Engine cloud service so as to
elevate scalability and reliability of the application. By using cloud computing from a credible service provider, the
application aims to minimize down-time and bottleneck comparing to self-hosting service on a single computer. In
addition, hosting application on a cloud platform also enables the application cost to scale better with the number of
users and data traffic since the service provider offers multiple pricing tiers for different amount of traffic. Hence,
the development stage which does not require large volume of transactions can make use of the free tier and scale up
once the application acquires certain user base and transaction volume. Moreover, Datastore database provided by Google
also supports caching and reliably distributes data to different locations with minimal latency.  Regarding
implementation of the server program, since client and server communicate through a RESTful API, the server can also be
implemented in any programming languages. In this project, the web server is implemented in Java with Google Endpoints
API \cite{Cloud Endpoints} with Cloud Search API \cite{SearchAPI}.  This approach helps promote code reuse by sharing
Java object between client and server codebase. In the future, the server can be swapped out by any web server if there
is need for finer control or performance improvement.

\subsection{Database}
Google Cloud Datastore is chosen to be the database backend of the system to reliably store and deliver data. Google
Cloud Datastore is a schema-less NoSQL Datastore providing robust, reliable storage for the web application. In
contrast with SQL-like database, Datastore treats each record in the database as a key-value pair without unnecessary
indexes. Comparing to the traditional relation databases, the Datastore uses a distributed architecture to automatically
managing scaling to very large data sets. Additionally, Datastore is hosted as part of the Google App Engine platform,
hence is fully managed with no planned down time by Google. Due to its different way of representing and managing data,
the Datastore can be easily scaled, allowing the application to maintain high performance as the traffic is increased.

\section{Implementation}
\subsection{Google Cloud Datastore}
\subsubsection{DataRecord}
The \texttt{DataRecord} class is used to contain the information of a user's reading values uploaded from the client
application or hardware. The model contains various properties about the data record including the type e.g. Simple,
Blood Pressure data, created date and the reading value represented by a String. The representation of a reading depends
on the type of interested measurement, hence the Datastore Entity does not enforce any constraint on the String format
and relies on the client and server application to correctly serialize and deserialize the value from and to the correct
string. \texttt{DataRecord} has the child - parent relationship with a user entity that supports efficient querying of
all instances that belong to a given user. Lastly, the type attribute is marked to be indexed by Datastore engine so as
to the server can query on specific type of data of interest.
\begin{minipage}{\linewidth}
\lstset{language=java,caption={DataRecord properties},label=DataRecord properties}
\begin{lstlisting}
@Entity
public class DataRecord {
    @Id public Long id;
    @Parent private Ref<HealthDroidUser> user;
    private Date date;
    @Index private int type;
    private Date createdAt;
    private String value;
}
\end{lstlisting}
\end{minipage}

\subsubsection{HealthDroidUser}
The \texttt{HealthDroidUser} class is used to represent a user of the application separately from their Google account.
The \texttt{HealthDroidUser} instance is created when the user logs into the system for the first time and will be used
as the reference for all their relevant data. The user object will not be created again on the next time the user logs
in.

\subsubsection{RegistrationRecord}
The \texttt{RegistrationRecord} class is used to represent a device that logged into the application and ready to receive
pushed message via Google Cloud Messaging.  The \texttt{RegistrationRecord} instance is created when a device logs into the
system and removed from Datastore database if it fails to send the message to the device i.e. the device is no longer
used or signed in with another user.

\subsubsection{SubscriptionRecord}
The \texttt{SubscriptionRecord} class is used to represent a subscription relationship between a user (subscriber) and
another user (target) and create a many-to-many relationship among users. The \texttt{SubscriptionRecord} is created
upon receiving a request to subscribe from the client application and deleted upon receiving a request to unsubscribe.
Since such a request need to be approved by the subscribed party, the \texttt{SubscriptionRecord} contains a
\texttt{isAccepted} value to indicate the status, which is false by default and only set to true once the user accepts
that. \texttt{SubscriptionRecord} has a child-parent relationship with the target user, and a reference to the
subscriber. Thus, the instances can be queried when finding users who subscribed to a given users or users subscribed to
by a given user.

\begin{minipage}{\linewidth}
\lstset{
    frame=tb,
    caption={SubscriptionRecord},
    label=SubscriptionRecord
    language=Java,
    aboveskip=\baselineskip,
    belowskip=\baselineskip,
    showstringspaces=false,
    columns=flexible,
    basicstyle={\small\ttfamily},
    numbers=none,
    breaklines=true,
    breakatwhitespace=true,
    tabsize=3
}
\begin{lstlisting}SubscriptionRecord
@Entity
public class SubscriptionRecord {
    @Id Long id;
    @Index private Boolean isAccepted;
    @Index private transient Ref<HealthDroidUser> subscriber;
    @Parent private transient Ref<HealthDroidUser> target;
}
\end{lstlisting}
\end{minipage}

\subsection{Server side}
The application server is implemented with Google Cloud Endpoints API in Java. The server can be accessed through a
RESTful API via HTTP requests and responses. In this project, several models of the application can be interacted with
through separate classes called an endpoint, each represent a feature and relevant methods. Each endpoint consists of
one or more methods annotated with \texttt{@ApiMethod} annotation and is exposed to outside accesses.

\subsubsection{Data}
\texttt{Data API} is contained in \texttt{DataEndpoint}. The Data API contains the following methods \\
\begin{minipage}{\linewidth}
\lstset{
    frame=tb,
    caption={Data API},
    label=Data API
    language=Java,
    aboveskip=\baselineskip,
    belowskip=\baselineskip,
    showstringspaces=false,
    columns=flexible,
    basicstyle={\small\ttfamily},
    numbers=none,
    breaklines=true,
    breakatwhitespace=true,
    tabsize=3
}
\begin{lstlisting}
@ApiMethod(name = "add")
public DataRecord addData(@Named("value") String value, @Named("date") Date date, @Named("type") int type, User user)
@ApiMethod(name = "get");
public List<DataRecord> getDataRecord(@Nullable @Named("userId") String userId, @Nullable @Named("after") Date after)
\end{lstlisting}
\end{minipage}

\begin{table}
\begin{center}
    \begin{tabularx}{\textwidth}{|l|X|}
        \hline Method & Description \\
        \hline addData() & 
            this method is used to add a data entry to datastore. The method is only allowed to be access by
            authenticated user as required in user parameter \\
        \hline getDataRecord() &
            this method is used to get the data from datastore. The method takes 2 optional arguments namely userId, to
            specify the user which data entries belong to and after, to set a lower limit of date in data entries. The
            "after" argument can be used to efficiently get only the new data entries in datastore for update operation
            \\
        \hline
    \end{tabularx}
\end{center}
\caption{Data API}
\end{table}

\subsubsection{User}
User API provides methods to interact with the user model of the application.
\begin{minipage}{\linewidth}
\lstset{
    frame=tb,
    caption={User API},
    label=User API
    language=Java,
    aboveskip=\baselineskip,
    belowskip=\baselineskip,
    showstringspaces=false,
    columns=flexible,
    basicstyle={\small\ttfamily},
    numbers=none,
    breaklines=true,
    breakatwhitespace=true,
    tabsize=3
}
\begin{lstlisting}
@ApiMethod(name = "add")
public HealthDroidUser addUser(User user)

@ApiMethod(name = "get")
public List<HealthDroidUser> getUser(@Nullable @Named("userId") String userId)

@ApiMethod(name = "query")
public List<HealthDroidUser> queryUser(@Named("queryString") String queryString)
\end{lstlisting}
\end{minipage}

\begin{table}
\begin{center}
    \begin{tabularx}{\textwidth}{| l | X |}
        \hline Method & Description \\
        \hline addUser() & 
            this method is used to add a user to datastore. The method is only allowed to be access by
            authenticated user as required in user parameter. Upon user signing in, the method is called and user is
            created if not already existing in the database. The method is idempotent and being called multiple times
            does not recreate the user. In addition, the method also create an entry used by Google Search API to
            support searching with Google's API.  \\
        \hline getUser() &
            this method is used to get the user from datastore. The method takes a userId as String and return
            information about the data from the database \\
        \hline queryUser() &
            this method is used to query a list of users from part of the user ID. Since Datastore does not index all
            attributes by default and hence not allow to straight forward search for user ID string. To support the
            query for large number of users in the database, the method uses Google Search API. \\
        \hline
    \end{tabularx}
\end{center}
    \caption{User API}
\end{table}

\subsubsection{Registration}
\begin{minipage}{\linewidth}
\lstset{
    frame=tb,
    caption={Registration API},
    label=Registration API
    language=Java,
    aboveskip=\baselineskip,
    belowskip=\baselineskip,
    showstringspaces=false,
    columns=flexible,
    basicstyle={\small\ttfamily},
    numbers=none,
    breaklines=true,
    breakatwhitespace=true,
    tabsize=3
}
\begin{lstlisting}
@ApiMethod(name = "register")
public void registerDevice(@Named("regId") String regId, User user)

@ApiMethod(name = "unregister")
public void unregisterDevice(@Named("regId") String regId)

@ApiMethod(name = "listDevices")
public CollectionResponse<RegistrationRecord> listDevices(@Named("count") int count)
\end{lstlisting}
\end{minipage}

Registration API provides the methods to register and unregister a device to the backend system for receiving
notifications via Google Cloud Messaging.

\begin{table}
\begin{center}
    \begin{tabularx}{\textwidth}{| l | X |}
        \hline Method & Description \\
        \hline registerDevice() & 
            this method is used to register a device to the backend. The method is only allowed to be accessed by an
            authenticated user and create a \texttt{RegistrationRecord} entity. \\
        \hline unregisterDevice() &
            this method is used to unregister a device to the backend.\\
        \hline listDevices() &
            this method is used to query a list of all device registrations that belong to a user. \\
        \hline
    \end{tabularx}
\end{center}
\caption{User API}
\end{table}

\subsubsection{Subscription}
Subscription API provides the methods to interact with subscriptions between users in the datastore. \\
\begin{minipage}{\linewidth}
\lstset{
    frame=tb,
    caption={Subscription API},
    label=Subscription API
    language=Java,
    aboveskip=\baselineskip,
    belowskip=\baselineskip,
    showstringspaces=false,
    columns=flexible,
    basicstyle={\small\ttfamily},
    numbers=none,
    breaklines=true,
    breakatwhitespace=true,
    tabsize=3
}
\begin{lstlisting}
@ApiMethod(name = "subscribe")
public SubscriptionRecord subscribe(@Named("target") String target, User user)

@ApiMethod(name = "accept")
public SubscriptionRecord acceptSubscription(@Named("subscriptionId") Long subscriptionId, User user)

@ApiMethod(name = "list")
public List<SubscriptionRecord> listSubscriptions()

@ApiMethod(name = "subscribed")
public Collection<SubscriptionRecord> getSubscribed(User user)

@ApiMethod(name = "subscribers")
public Collection<SubscriptionRecord> subscribers(@Named("userId") String userId) 

@ApiMethod(name = "pending")
public Collection<SubscriptionRecord> pending(@Named("userId") String userId)

@ApiMethod(name = "unsubscribe")
public List<SubscriptionRecord> unsubscribe(@Named("userId") String userId, @Nullable @Named("target") String target)
\end{lstlisting}
\end{minipage}

\begin{table}
\begin{center}
    \begin{tabularx}{\textwidth}{| l | X |}
        \hline Method & Description \\
        \hline listSubscription() & 
            this method is used to list all the subscription records in the database. \\
        \hline getSubscribed() & 
            this method is used to get all the users that is subscribed to by a give user. \\
        \hline subscribers() & 
            this method is used to get all the users that subscribed to a given user. \\
        \hline subscribe() & 
            this method is used to subscribe to a user. \\
        \hline unsubscribe() & 
            this method is used to unsubscribe from a user. \\
        \hline
    \end{tabularx}
\end{center}
\caption{Subscription API}
\end{table}

\subsection{Client side}

\subsubsection{Overall design}
The android application is implemented with the following components:
\begin{enumerate}
    \item SignInActivity that handles signing in of users
    \item MainActivity that serve as the hosting activity for different views of the application with a Navigation
        drawer panel to allow users to switch among the views
    \item Google Cloud service that handles the Google Cloud Messaging communication to listen to pushed messages from
        the servers and redirect them to appropriate components
    \item Database services that provides method to access the logic models uses in the class
\end{enumerate}

Each view of the application is implemented in a class called Fragment.  Navigations between parallel fragments are
achieved through Navigation Drawer as suggested by Google's new Material android design guidelines
\cite{NavigationDrawerPattern}. Each view is implemented as an independent fragment so that the user interface of the
application can be refactored and reimplemented without changing the view itself.  Transitioning between fragments is
handled by the Main activity's fragment manager. The Main Activity also acts as the main communication among 
fragments or between a service and fragments. The communication mechanism follows Observer pattern
\cite{Gamma:1995:DPE:186897} with each fragment registers and unregisters itself with the activity for interested
events.

\subsubsection{Local database}
In order to interact with the local database, all views in the application use centralized point of access. The client
application logic business requires 2 models \texttt{Data} and \texttt{User} and hence there are 2 contract classes that
represent the interface to the database. These interfaces effectively simplify the dependencies and interaction between
the views and the models. By having a single mean of access, it is easier to change the way the data is access globally
by changing the interface and also make it easier to debug if necessary.

\subsubsection{Utility Services}
The application consists of a set of Service class extending Android's IntentService that handle interactions with the models
and the application server. The service is invoked from any of the views by creating an Intent with appropriate
arguments and the result is then propagated to all effective views that are interested in the result. Once finishing
servicing a request, an event is created and broadcasted to the main activity. Since there are more and one view that
are interested in a given model, the event is received by the main activity and then propagated to its current view for
displaying the result if the view is registered with the activity. Upon creating and replacing a fragment to the
activity, a fragment is registered to receive the broadcasted events following Observer pattern. Each event is
implemented with a pair of Publisher-Listener interfaces with the view implementing the listener interface and the main
activity implementing the according publisher interface.
The service classes run on a separate thread rather than UI thread hence allows calling network operations in the
service implementation. In this application, each request is translated to one or more network operations to the back
end server and executed in an AsyncTask.

\paragraph{Subscription Service} \mbox{} \\
The \texttt{SubscriptionService} is used to handle the interaction with the users and subscriptions. The service provides the
methods to fetch a list of users of the application, subscribe to a user, cancel a subscription request and accept a
subscription requests by other users.  In local domain, the subscriptions from a logged in user to other users can be
represented as a status attached to a User entity indicating not-subscribed, subscribed or pending request. The states
of the subscription request is described in a Finite State Machine diagram. The aforementioned status is also
implemented in the views to change the widgets accordingly so as to reflect the status of the subscription.

\begin{figure}[!ht]
    \caption{Subscription model FSM}
    \centering
    \includegraphics[width=1\textwidth]{subscription_fsm}
\end{figure}

\paragraph{Data Service} \mbox{} \\
The \texttt{DataService} is used to handle the interaction with the data records in the database. In the client
application, \texttt{DataService} consists of 1 method to fetch data from the back end server and store in local SQLite
database. The fetching is performed in rolling basis to reduce the network traffic i.e. only the new data if fetched.
Each subscribed user entry in the local database maintains a \texttt{lastUpdated} field to indicate the latest time it
was updated.  This information is initialized to a zero day when the user is created and updated every time the data of
that user is fetched from the database. Upon unsubscription, the entry is cleared and reassigned to zero day so that all
data of that user is fetched if the user is subscribed to again. During the fetch and insert to database operation, the
list of data entries is iterated and the latest date is recorded and updated for a given user.

\subsubsection{Google Cloud Messaging}
Google Cloud Messaging (GCM) is the free service provided by Google to allows developers to send messages between
servers and client applications. The service supports both upstream and downstream messages but only downstream message
is used in this application.
To integrate the application with the Google Cloud Messaging service, each instance of the application must register
itself and implement a listener service to listen for the pushed message \cite{CloudMessaging}. Registration to the
backend is invoked through a \texttt{RegistrationIntentService}. In the recent versions of Google Cloud Messaging
library for Android, GCM register() has been deprecated and replaced with \texttt{InstanceID} hence InstanceID API is used in
this application to keep it up to date with the latest change. Upon receiving the \texttt{InstanceID}, the service also
invokes the registration to the application server (Endpoint component) to register its token which is used to identify
the device.
Receiving of downstream message is handled by subclasses of \texttt{GcmListenerService}. This is the receiving point
that process the messages sent from the application server to signal changes of data or subscription e.g. request
accepted events. The GCM listener service must be implemented so that it agrees on the format and content of the message
sent by the server module.

\subsubsection{Graph display}
Displaying of graph is contained in GraphFragment class. This fragment also allows users to choose various ways of
representation of the chart such as which user, time range to display as well as appropriate style of chart for
different types of data. Displaying of the widget is handled by MPAndroidChart by Philipp Jahoda \cite{MPAndroidChart},
provided as an open source library on GitHub under Apache License, Version 2.0. The chart placeholder is populated with a
\texttt{Chart} subclasses and then appropriate methods are called depending on the choice of representation.

\paragraph{Graph Fragment}\mbox{} \\
\texttt{GraphFragment} and its subclasses are in charge of create the Chart and inflate it to the layout container.
\texttt{GraphFragment} can be subclasses for specific type of data. In the scope of this project, it is implemented in 2
subclasses namely \texttt{SimpleDataFragment} that uses line chart with 1 entry per each x value and
\texttt{BloodPressureFragment} that uses scatter chart with 2 entries per each x value. The subclasses are required to
implement the abstract methods in \texttt{GraphFragment} to display the data accordingly.  The \texttt{GraphFragment}
contains the following abstract methods that are implemented in the concrete classes:

\begin{figure}[!ht]
    \caption{GraphFragment class diagram}
    \centering
    \includegraphics[width=1\textwidth]{GraphFragment}
\end{figure}

\begin{lstlisting}
abstract String getDescription();
abstract Cursor getQuery(HealthDroidDatabaseHelper);
abstract Chart makeChart(Context);
abstract DataPool makeDataPool();
\end{lstlisting}

With the appropriate dependencies constructed in the concrete classes, an asynchronous task extending from
\texttt{DisplayDataTask} will then be invoked by on the time range selection and inject the dependencies dynamically.
Actual invocation of the \texttt{Chart} widget is handled by a \texttt{ChartAdapter} object. However, each type of
chart provided by MPAndroidChart has different attributes and hence is handled differently. The \texttt{GraphFragment}
class only has dependency on abstract class \texttt{ChartAdapter} and relies on the concrete implementations to provide
the correct way of interaction with the chart. Constructing of the \texttt{ChartAdapter} is declared as abstract in
GraphManager and implemented by the its concrete subclasses, with each type of chart creating its suitable chart
adapter. In the scope of this project implementation, the \texttt{Simple} creates its \texttt{LineChartAdapter} for
inserting data to Line chart while the \texttt{BloodPressureFragment} creates its \texttt{ScatterChartAdapter} for
inserting data to Scatter chart. Thus, when the application is extended for new type of graph, the new concrete class
must create its own Chart adapter and changing to existing code is not required.

\paragraph{KeyCreator}\mbox{} \\
The \texttt{KeyCreator} class has the responsibility of creating the values on x-axis as well as the lookup key in the
resulted dictionary. The \texttt{KeyCreator} is declared as an interface and implemented by concrete classes namely
\texttt{ByDayKeyCreator}, \texttt{ByWeekKeyCreator} and \texttt{ByMonthKeyCreator}

\begin{table}
\begin{center}
    \begin{tabularx}{\textwidth}{| l | X |}
        \hline Method & Description \\
        \hline createKey() & 
            this method takes a date string in RFC3339 format as parameter and returns the corresponding key string in
            the correct format \\
        \hline getDateRange() &
            this method takes the first date and the last date from the data set i.e. the range of values on x-axis of
            the chart and returns a wrapper object that contains these 2 values and normalize the range to be at least
            10 units. \\
        \hline getTimeUnit() &
            this method takes no arguments and return the corresponding value from Java Calendar to help get the value
            from GregorianCalendar \\
        \hline
    \end{tabularx}
\end{center}
    \caption{KeyCreator methods description}
\end{table}

\paragraph{DataPool}\mbox{} \\
As the graph fragment is required to display data grouped by date differently based on users' choices, the raw data
entries acquired from database querying need to be processed and the accumulated value of a day, week or month is
computed for displaying to the users. For each combination of choices provided by the users, the graph contains
different number of entries on x-axis and the different corresponding y-value. To achieve that, every time the task is
invoked, all data must be queried and prepared for processing in a \texttt{DataPool} object. The \texttt{DataPool}
object acts as a central repository for data of all users in the local database to be grouped by their date and then
creates a Map containing the grouped result. The \texttt{DataPool} class contains a mapping between a username and their
relevant data entries as well as the dictionaries mapping date to values. The resulted dictionaries contain appropriate
keys generated by a KeyCreator object for each date range selection e.g. days (05/03/2015) for "by day" choice and month
(12/2015) for "by month" choice.

\begin{figure}[!ht]
    \caption{DisplayDataTask}
    \centering
    \includegraphics[width=1\textwidth]{DisplayDataTask}
\end{figure}

\begin{table}
\begin{center}
    \begin{tabularx}{\textwidth}{| l | X |}
        \hline Method & Description \\
        \hline accumulate() & 
            this method starts the accumulation of data to create a map from user name to their data set contained in an
            object \texttt{DataAccumulator} \\
        \hline insertToChart() &
            this method takes a \texttt{ChartAdapter} as argument to insert the prepared data from accumulate() into a
            chart \\
        \hline
    \end{tabularx}
\end{center}
\caption{DataPool methods description}
\end{table}
\noindent \texttt{DataPool} depends on 2 interfaces namely \texttt{KeyCreator} and \texttt{DataAccumulator}. It can be
injected with any subclass of \texttt{KeyCreator} depends on the date range required.  Once invoked, the
\texttt{DataPool} object iterates through all users in local database and creates a \texttt{DataAccumulator} instance
for each user. Since there are different types of DataPool for each type of chart, the \texttt{DataPool} class is
declared as abstract and requires its subclasses to construct the correct \texttt{DataAccumulator} instance. 

\paragraph{DataAccumulator}\mbox{} \\
The \texttt{DataAccumulator} class has the responsibility of containing a mapping from a string representing a date to a
data representation. Depends on different type of data, the representation is constructed differently. In the scope of
this project, DataAccumulator is extended by 2 concrete classes namely \texttt{SimpleDataAccumulator} that has 1
numerical value per date key and \texttt{BloodPressureDataAccumulator} that has 2 numerical values per date key. The
\texttt{DataAccumulator} exposes the method \texttt{accumulate(String value, String key)} in which value is the String
representation of the data and the concrete class is responsible for deserialize the string to achieve numerical values.
\texttt{DataAccumulator}
\texttt{DataAccumulator} has dependency on the \texttt{KeyCreator} class and required an instance of \texttt{KeyCreator}
in its constructor. The caller of \texttt{DataAccumulator} class is required to provide the suitable \texttt{KeyCreator}
instance for its need.

\subsubsection{User Display}
\paragraph{User RecyclerView} \mbox{} \\
In recent versions of Android Support Library, Google has released the \texttt{RecyclerView} as a flexible view for
providing a limited window into a large data set. This widget is a more advanced and flexible implementation of
\texttt{ListView} that used to be the standard way to display a list of items. \texttt{RecyclerView} contains the view
to display items that can be scrolled very efficiently by maintaining a limited number of views\cite{RecyclerView}.
\begin{figure}[h]
    \centering
    \includegraphics[width=\textwidth]{RecyclerView}
    \caption{Recycler View structure}
\end{figure}
A recycler view delegate its responsibility to \texttt{LayoutManager} for displaying the layout and \texttt{Adapter} to
map an object model to a view. When an adapter is created, the object representation is bound to the view contained in
the corresponding \texttt{ViewHolder}. By delegating the responsibility to handle the binding and interaction with the
views to the programmer, RecyclerView effectively reduces the overhead and unnecessary work in the UI.

\paragraph{User ViewHolder} \mbox{} \\
The view for a user is managed by \texttt{UserViewHolder} class. This view contains the information about the users a
button that allows to subscribe, unsubscribe depending on the current state. Interactions on the button is delegated to
the event listener set by the client of this class i.e. UserFragment. Hence, interactions on the view is propagated
upstream to the fragment to the database model and change of the model is listened by the fragment and propagated to the
view to change accordingly. Thus, series of callbacks implemented by 2 interfaces \texttt{SubscriptionChangeListener} and
\texttt{SubscriptionChangePublisher} created a 2-way binding of models in the application.

\begin{figure}[!ht]
    \caption{User view class diagram}
    \centering
    \includegraphics[width=1\textwidth]{UserView}
\end{figure}
\clearpage
\section{Discussion}
\subsection{Android development practices}
In order to approach the latest change in new Android version and Android SDK, the project aims to use the latest tools
and practices suggested by Google. At the time this report was written, the libraries used by the application is updated
the latest version available so that it can better keeps up with the trend and current trend. The project also uses the
most up-to-date libraries and classes provided by Google in both Android application and backend server so that it has
better performance and better chance for further development effort.
For future work, the design and layout of the application maybe changed to suit Google's new design guideline as new
version of android is released. Additionally, due to time constraint, only recent version of Android (from Android 5.0)
is targeted and hence future work can be done to extend the application to support older versions.

\subsection{Software engineering practices}
The client side application was made with consideration about design and components choices. Different standards in
software engineering was considered and followed throughout the development of the application so that the code can be
read and reused by other developers. Serious consideration and effort has been put in the designing the graph components
such that the application can be easily extended to other types of data i.e. blood pressure, blood glucose etc. as well
as different ways of viewing data in the future. As the graph feature of the application is expected to be flexible and
supports various options of displaying, the responsibility was split into small classes with single responsibility.
Most components depend only on interfaces and hence new classes complying to the interface can be created and easily
plugged into the code base with dependency injection. Otherwise, there would have been lots of code duplications in and
type checking to ensure that the application works as expected. Without careful design, the code would have been tightly
coupled and number of classes increase significantly as each class handles a specific combination of the choices. As the
project proceeds, the unstructured code base could grow beyond maintainability.

\section{Conclusion}
In conclusion, this project has implemented a prototype of cloud-based medical data managing solutions. The application
implemented in this application aims to provide a reliable platform for large volume of transactions. The application
consists of an Android application and a backend server that runs on Google App Engine which is an efficient and scalable
approach for server and database hosting. The projects provided working application as a proof of concept for a full
end-to-end system that allows the users to post, retrieve and share their data. The work of this project also suggested
a potential expansion to integrate with the real measuring devices so that recorded data can be submitted to the
database in real time. Under careful consideration of the architecture and class design, the application can be easily
extended to support more types of data as well as displaying options. Moreover, this project has targeted the latest
libraries and android design guidelines by Google so that it can stay more up to date and maintainable in future work.
\vfill

\bibliography{fyp} 
\bibliographystyle{ieeetr}
\end{document}

